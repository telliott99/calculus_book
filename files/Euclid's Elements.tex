\documentclass[11pt, oneside]{article} 
\usepackage{geometry}
\geometry{letterpaper} 
\usepackage{graphicx}
	
\usepackage{amssymb}
\usepackage{amsmath}
\usepackage{parskip}
\usepackage{color}
\usepackage{hyperref}

\graphicspath{{/Users/telliott/Github/calculus_book/png/}}
% \begin{center} \includegraphics [scale=0.4] {gauss3.png} \end{center}

\title{Euclid's Elements}
\date{}

\begin{document}
\maketitle
\Large

In this chapter we will study nine \emph{Propositions} from the first volume Euclid's \emph{Elements}.  It was put together as a compendium of geometry for students.  One thing we will see is how the propositions build on one another.  

The first three propositions are \emph{constructions}, e.g. the very first asks us to construct a triangle with all three sides equal, an equilateral triangle.

\subsection*{Prop. I.1}
To construct an equilateral triangle on a given line segment.
\begin{center} \includegraphics [scale=0.4] {PI_1a.png} \end{center}

The tools we have are a straightedge and a compass.  The compass is collapsible, meaning that it cannot be used to transfer distances since it loses its setting when lifted from the page.  As we'll see in the next part, this is a problem with a solution.

Euclid was smart enough to know about compasses and how to set them.  The idea he had was:  to make the fewest possible assumptions, and a non-collapsible compass was a luxury he didn't need, since he could accomplish the same end without it, as we will see.

The first step is to draw two circles on centers $A$ and $B$.
\begin{center} \includegraphics [scale=0.4] {PI_1b.png} \end{center}

The circles are drawn with each radius equal to the line segment $AB$.  It is a property of circles that all points on the circle are at the same distance from the center.  Thus all points on the left-hand circle are equidistant from $A$, and all points on the second one are equidistant from $B$.  

Therefore, the point $C$  where the circles cross is equidistant from \emph{both} $A$ and $B$ (there is another such point at the bottom---in fact, if we drew the line segment from that point to $C$, it would be the perpendicular bisector of $AB$).

For this, we don't really need the entire circles, just the part where the arcs cross at $C$.

\begin{center} \includegraphics [scale=0.4] {PI_1c.png} \end{center}

Now use the straight edge to draw $\triangle ABC$.  Since $AC = AB$ and $BC = AB$, we know that $AC = BC$.  The triangle is equilateral.

We put a little box to show that the proof is complete.

$\square$

The proof doesn't stand on its own.  We used one definition (D) and a common notion (CN).

$\circ$ \ D I.15  all radii of a circle are equal.

$\circ$ \ CN I.1  things which equal the same thing also equal one another.


\subsection*{Prop. I.2}
To place a straight line equal to a given straight line with one end at a given point.

We will construct a line segment at $A$ equal in length to $BC$ (left panel).  The first thing is to draw the line segment $AB$ and construct an equilateral triangle on it (right panel).   
\begin{center} \includegraphics [scale=0.4] {PI_2a.png} \end{center}

We know how to do this (from $P I.1$).  

Next, construct a circle on center $B$ with radius $BC$ and extend the line segment $DB$ to point $G$.  

Then, construct a circle on center $D$ with radius $DG$ and extend $DA$ to that circle at point $L$.  

We have:

\begin{center} \includegraphics [scale=0.4] {PI_2b.png} \end{center}

As common radii of the circle on center $B$, we have $BC = BG$.  

As common radii of the circle on center $D$, we have $DL = DG$.  

As sides of an equilateral triangle, we have $DA = DB$.

We use CN $I.3$:  if equals are subtracted from equals, then the remainders are equal.  Thus, $AL = BG$.  But we had above that $BC = BG$.  Therefore, $AL = BC$, by CN $I.1$.  

Q.E.D. or "quod erat demonstrandum", which is Latin, and in the original Greek \emph{the very thing it was required to have shown.}

$\square$

Note in passing, the orientation is determined by $AB$.  We have not shown how to transfer the length with an arbitrary orientation.  We will solve this next.

\subsection*{Prop. I.3}
To cut off the lesser of two unequal straight lines from the greater.

\begin{center} \includegraphics [scale=0.4] {PI_3a.png} \end{center}

In the left panel, we have the line segment $AB$ and a smaller one just labeled $C$.  To do the construction, use the method of P $I.2$ and transfer $C$ to point $A$, forming $AD$.  

Next, use $AD$ as the radius of a circle on center $A$.  Then, $AE = AD$, but $AD = C$.  Hence $BE = AB - C$ as required.

$\square$

At this point, we have a method to mark off a given length from a larger length, even though all we have is a collapsing compass.  Therefore, going forward, we can act as if we have a standard compass, that holds its setting after being lifted from the paper.

We also have the means to an important \emph{trichotomy}.  Comparing two line segments, one of three things is true:  the first is smaller than the second, they are equal, or the second is smaller than the first.

\subsection*{Prop. I.4}

If two triangles have two sides equal to two sides respectively, and have the angles contained by the equal straight lines equal, then they also have the base equal to the base, the triangle equals the triangle, and the remaining angles equal the remaining angles respectively, namely those opposite the equal sides.

\begin{center} \includegraphics [scale=0.4] {PI_4a.png} \end{center}

This is not a construction, unlike the previous three propositions.  It is a method for proving congruence (equality) of two triangles 
\[ \triangle ABC \cong \triangle DEF \]

Elsewhere in this book we would call the method SAS or \emph{side angle side}.  Given that $AB = DE$ and $AC = DF$ and that the angles between them at the vertices $A$ and $D$ are also equal, the two triangles are congruent:  all three angles and all three sides are equal.

This is a proof that SAS is correct.

The proof is by superposition.  The facts establish the positions of the points $B$ and $C$, which determines $AB$ and so the angles at vertices $B$ and $C$.

Euclid says that if we lift up $\triangle ABC$ and lay it on top of $\triangle DEF$ then $B$ coincides with $E$ and $C$ coincides with $F$ so $BC = EF$.

$\square$

This seems perhaps a little shaky logically, and it's not a method of proof that Euclid uses much.

But one might instead have taken this proposition as a postulate.  The source, above, says that David Hilbert claims that under the hypotheses of the proposition it is true that the two base angles are equal, and then proves that the bases are equal.

We have used SAS to prove SSS, that all three sides are equal.

In any event, SAS is very commonly used to prove congruence.  
\begin{center} \includegraphics [scale=0.4] {SAS.png} \end{center}

In this diagram, sides of equal length are indicated by one or more hash marks.  Equal angles are indicated by dots (another common method is to draw an arc with a hash across it).

The other methods for proving congruence use two equal angles and a side.  Two equal angles imply the third angle is also equal (since they add to a half-circle or 180 degrees), so the two triangles are similar.  To prove they are congruent, It is important that the equal sides are flanked by the same angles, or equivalently, are opposite the same angle.

These methods using two angles are referred to as ASA
\begin{center} \includegraphics [scale=0.4] {ASA3.png} \end{center}

 and AAS.
\begin{center} \includegraphics [scale=0.4] {AAS.png} \end{center}

The next proposition refers to triangles with two sides equal, referred to as isosceles.

\subsection*{note on notation}

The Greeks, including Euclid, adhere to certain conventions.  For example, points are always labeled with letters, line segments are referred to by the endpoints, and angles by the line segments that determine them.

I don't know about you but I find myself tracing out angles from the three points, again and again.

Let us try a different approach.  We could give labels to the angles like $\alpha, \beta \dots$, and to the sides opposite vertices as $a$ opposite $A$ and so on.  

But let's go with something even more dramatic.  Dispense with labels altogether and use colored dots for equal angles and colored bars for equal lengths.  We repeat the famous proof of Thales' theorem from Euclid's \emph{Elements}.

\subsection*{Prop. I.5}

In isosceles triangles the angles at the base equal one another, and, if the equal straight lines are produced further, then the angles under the base equal one another.

In what follows, all the pieces are with reference to the initial construction, first figure, below.

\begin{center} \includegraphics [scale=0.6] {PI_5d.png} \end{center}

\begin{center} \includegraphics [scale=0.6] {PI_5e.png} \end{center}

\begin{center} \includegraphics [scale=0.6] {PI_5f.png} \end{center}

\begin{center} \includegraphics [scale=0.6] {PI_5g.png} \end{center}

$\square$

Euclid's proof of the converse is short and introduces the method of contradiction, or \emph{reductio ad absurdum}.  That is the next proposition.
  
\subsection*{Prop. I.6}

If in a triangle two angles equal one another, then the sides opposite the equal angles also equal one another.

Suppose we have $\triangle ABC$ with $\angle ABC = \angle ACB$.

\begin{center} \includegraphics [scale=0.5] {PI_6a.png} \end{center}

If $AB$ does not equal $AC$, then one of them is greater.  Let $AB$ be greater, then cut off $DB$ from $AB$ such that $DB = AC$.

Since $DB$ equals $AC$, and $BC$ is common, therefore the two sides $DB$ and $BC$ equal the two sides $AC$ and $CB$ respectively, and the $\angle DBC$ equals $\angle ACB$. 

Therefore the base $DC$ equals the base $AB$, and $\triangle DBC$ equals $\triangle ACB$, the less equals the greater, which is absurd. Therefore $AB$ cannot be unequal to $AC$, 

It therefore equals it.

In other words, we use SAS to prove that triangle $\triangle ABC \cong \triangle DBC$.  But that is absurd, since the whole is greater than the part, $\triangle ABC$ cannot be equal to a part of itself.

Our original assumption that $AB$ does not equal $AC$ must be false.

$\square$

We will do three more.  They are short, sweet and powerful.

\subsection*{Prop. I.16}

In any triangle, if one of the sides is produced (extended), then the exterior angle is greater than either of the interior and opposite angles.

\begin{center} \includegraphics [scale=0.5] {PI_16a.png} \end{center}

The claim is that the exterior angle $e$ is greater than either of the interior angles:  $s$ or $t$.  (We know it is equal to their sum, but Euclid hasn't proved that yet).

Find the midpoint of the side opposite $s$ and draw the indicated line segment (right panel, below), so that the two segments marked with red arrows are equal, as well as the segments marked with black arrows.  
\begin{center} \includegraphics [scale=0.5] {PI_16b.png} \end{center}

By SAS and the vertical angle theorem, the two triangles with dotted lines are congruent, as indicated in the right panel by the labels on the angles.

But now clearly
\[ e + t > t \]
(the whole is greater than its parts) for $e$ plus the angle next to it, but this applies as well as the original interior angle $t$.

We can make a similar proof for angle $s$.

$\square$

\subsection*{Prop. I.18}

In any triangle, a greater side is opposite a greater angle.

\begin{center} \includegraphics [scale=0.5] {PI_18a.png} \end{center}

Given $b > a$, mark off $a$ on $b$.

\begin{center} \includegraphics [scale=0.5] {PI_18b.png} \end{center}

By the external angle theorem (I.16)
\[ v > t \]

But $v = s'$ (by isosceles $\triangle$, $I.5$) so 
\[ s' > t \]
And since $s > s' $
\begin{center} \includegraphics [scale=0.5] {PI_18a.png} \end{center}
\[ s > t \]

$\square$

We get the converse almost for free.

\subsection*{Prop. I.19}

In any triangle, a greater angle is opposite a greater side.

\begin{center} \includegraphics [scale=0.5] {PI_18a.png} \end{center}

We are given $s > t$ and want to prove $a < b$.  We proceed by considering the other possibilities.

It cannot be that $a = b$ because then $s = t$ by isosceles $\triangle$ ($I.5$), but we are given $s > t$.

So then suppose $a > b$.  By the previous proposition ($I.18$), we would have that $t > s$.  But this is again contrary to what we were given.  Hence $b > a$.

$\square$

We have made use of the trichotomy from before, that there are only three possibilities:
\[ a < b, \ \ \ \ \ \ a > b, \ \ \ \ \ \ a = b \]

This applies to line segments and angles as well as many other things.

The \emph{Elements} is wonderful, though challenging.  I think this is enough to give us a good taste of the basics of Greek geometry of lines and triangles, and methods of proof.  There is more to come:  Pythagoras, and circles and arcs of circles.

\end{document}