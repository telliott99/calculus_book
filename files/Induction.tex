\documentclass[11pt, oneside]{article} 
\usepackage{geometry}
\geometry{letterpaper} 
\usepackage{graphicx}
	
\usepackage{amssymb}
\usepackage{amsmath}
\usepackage{parskip}
\usepackage{color}
\usepackage{hyperref}

\graphicspath{{/Users/telliott/Github/calculus_book/png/}}
% \begin{center} \includegraphics [scale=0.4] {gauss3.png} \end{center}

\title{Induction}
\date{}

\begin{document}
\maketitle
\Large

\label{sec:Induction}

\subsection*{the problem}

Suppose we have some theorem that we \emph{think} might apply to all $k \le n$.  A classic example (Courant and Robbins) is:
\[ f(n) = n^2 - n + 41 \]
The function $f(n)$ produces a prime number for integer $0 < n < 41$, which is apparent by inspection (because any $n$ divides $n^2 - n$ and $41$ is prime).  But for $n=41$, the last two terms cancel, $41^2$ is divisible by $41$, thus the result cannot be prime.

Hamming has some other examples.  Here is one:
\[ f(n) = n(n-1)(n-2) \dots (n-k) \]
$f(n)=0$ for all $0 \le n \le k$, but will never be zero for any other $n > k$!  By choosing $k$ large, we can make the number of true cases as large as you like.

Furthermore, for any function $g(n)$, $f(n) + g(n)$ will have the same property.

\subsection*{proving a formula correct}

Later in the book, we will compute Riemann sums, and to do that we need to find formulas for the sum of integers, the sum of square integers, and so on.  

To keep it simple, let's start with a finite sum like the integers from $1$ to $n$
\[  1 + 2 + 3 + \cdots + n  \]

The numbers we seek are called the triangular numbers.  These are
\[ 1, 3, 6, 10, 15 \cdots \]

Somehow we have found a formula for the sum of the first $n$ integers, namely 

\[ 1 + 2 + \dots + n = S_n \]
\[ = \frac{n (n + 1)}{2} \]

Assuming the formula is correct for $S_n$, then it certainly follows that
\[ S_{n + 1} = \frac{(n)(n + 1)}{2} + (n+1) \]

Rearranging:
\[ = \frac{n(n + 1) + 2(n + 1)}{2} \]
\[ = \frac{(n + 1)(n + 2)}{2} \]

which is exactly what we'd get by substituting $n+1$ for $n$ in the formula.  So we have proven that if the $S_n$ formula is correct, then so is the one for $S_{n+1}$.

How do we know that $S_n$ is correct?

Check the \emph{base case}:
\[ S_1 = \frac{1(1 + 1)}{2} = 1 \]
Since $S_1$ is clearly correct, $S_2$ must be also, and then just continue all the way to $S_{n}$.
\[ S_1 \Rightarrow S_2 \Rightarrow \dots S_{n-1} \Rightarrow S_n \Rightarrow S_{n+1} \]
It must be true for \emph{every} integer $n$.

This is an example of an inductive proof in mathematics.  

We can visualize an inductive proof as a kind of chain.  We show that the base case is true, for some value of $n$.  Then we show that if the formula works for $n$ , it must work for $n+1$.

\begin{quote}Mathematical induction proves that we can climb as high as we like on a ladder, by proving that we can climb onto the bottom rung (the basis) and that from each rung we can climb up to the next one (the step).\end{quote}

- Graham, Knuth and Patashnik

[ There is a variant called \emph{strong} induction where we assume some statement is true for \emph{all} $0 < k \le n$. ]

\subsection*{other proofs}

Although we proved the formula already by induction, there's no harm in trying a different method.

There is a famous story about Gauss that, as a schoolboy, he "saw" how to add the integers from $1$ to $100$ as two parallel sums.

\begin{center} \includegraphics [scale=0.40] {gauss_sum.png}\end{center}
Added together horizontally, these two series must equal twice the sum of $1$ to $100$.  

But in the vertical, we notice that we have $n$ sums, each of which is equal to $n+1$.  So, again
\[ 2S = n (n+1) \]
\[ S = \frac{1}{2} \ n (n+1) \]
For $n=100$ the value of the sum is $5050$.  Another way of looking at this result is that between $1$ and $100$ there are $100$ representatives of the "average" value in the sequence, which (because of the monotonic steps) is $(100 + 1)/2 = 50.5$.  

Or alternatively, view the sum as ranging from $0$ to $100$ (with the same answer).  Now there are $101$ examples of the average value ($100 + 0)/2 = 50$).

Here is a striking \emph{visual proof} of the formula to obtain T$_n$, the $n^{th}$ such number.  The total number of circles in the figure below is $n \times (n+1)$ and this is exactly two times the sum of the integers from $1$ to $n$.

\begin{center} \includegraphics [scale=0.25] {sum_n.png}\end{center}
\[ 2S = n(n+1) \]

\subsection*{induction in geometry}

In the figure below is a polygon---an irregular heptagon.  Actually, there are three polygons altogether, there is the heptagon with $n+1$ sides, the hexagon with only $n$ sides that would result from cutting along the dotted line, and the triangle that is cut off.

We want to find a formula for the sum of the internal angles that depends only on the number of sides or vertices.

\begin{center} \includegraphics [scale=0.5] {polygon.png} \end{center}

The first part of the answer is to guess.  In the figure, you can see that by adding the extra vertex to go to the $n+1$-gon, we added a triangle, or perhaps you'd rather say than in going from $n+1$ to $n$ we lost a triangle.  

In either case, the difference is $180^\circ$.  The difference between having $n$ sides and $n+1$ sides is to add $180^\circ$.  

The second part of the argument is to suppose that $n=3$, in that case we must have simply $180^\circ$ degrees for a triangle.  So we guess that the formula may be
\[ S_n = (n-2)180^\circ \]
where $S_n$ is the sum of the angles in an $n$-gon.

We can use induction to prove that this formula is correct.

The proof has two parts.  We must verify the formula for a base case like the triangle, which we've done.  You may wish to check that it works for the square as well, but that's not strictly necessary.

The second part of the proof is to verify that in going from $n$ to $n+1$, we add another $180^\circ$.  \[ (n-2)180^\circ + 180^\circ \stackrel{?}{=} ((n+1)-2)180^\circ \]
On the left-hand side, we have the sum of angles for $n$ sides, which we assume is correct, and then we just add $180^\circ$ to it.  On the right, we have substituted $n+1$ into the formula.

Now we need to show that these are equivalent.  But of course
\[ (n-2)x + x = ((n+1)-2) x \]
\[ n - 2 + 1 = n + 1 - 2 \]
$\square$

Other examples:

\subsection*{sum of digits and divisibility}

It is very easy to check whether any number $n$ is divisible by $9$.  Simply add up all the digits:
\[ 234783738 \Rightarrow 2 + 3 + 4 + 7 + 8 + 3 + 7 + 3 + 8 \]
\[ = 5 + 11 + 11 + 10 + 8 \]
\[ = 16 + 21 + 8 \]
\[ = 7 + 3 + 8 = 9 \]
Yes, $9|234783738$.

We propose that

$9 | (10^n - 1)$ for all integers $n \ge 0$.

Suppose we know that $9 | 10^k - 1$ for some $n$. We mean that
\[ 10^k - 1 = 9x \]
for some $x$.  Multiply by $10$:
\[ 10 \cdot (10^k - 1) = 10 \cdot 9x \]
\[ 10^{k+1} - 10 = 9 \cdot 10x \]
\[ 10^{k+1} - 1 = 9 \cdot 10x + 9 = 9(10x + 1) \]
The right-hand side is clearly divisible by $9$, and then so is the left-hand side.

The base case is $9|0$ which is true by definition but may be confusing.  Try $n=1$, then $9|(10 -1)$ is certainly correct.

$\square$

Given this, it is easy to show that the sum of digits method always works.  I'll leave it as an exercise.

\subsection*{Odd number theorem}

Here is a simple but very useful inductive proof.

The \emph{odd number theorem} says that the sum of the first $n$ odd numbers is equal to $n^2$.  Here is a "proof without words".

\begin{center} \includegraphics [scale=0.4] {odd_number_theorem.png} \end{center}

We prove this by induction.

\[ \ (0 \times 2 + 1) +  (1 \times 2 + 1) + (2 \times 2 + 1) + (\dots \]
\[ \dots + (n-1) \times 2 + 1) = n^2 \]

Notice that the $n$th odd number is $2 \times (n-1) + 1$.

Our formula says that
\[ 1 + 3 + 5 + \dots + (2n - 1) = n^2 \]
If you like the summation style:
\[ \sum_{k=0}^n 2k - 1 = k^2 \]

As an example, the first five odd numbers are
\[ 1 + 3 + 5 + 7 + 9 = 25 = 5^2 \]

So, if we consider the next odd number, $n$ changes to $n+1$.  The left-hand side gets another term:  we add $2 \times (n+1)-1$ to it.  That is equal to $2n + 1$.

To maintain the equality, add the same quantity to the right-hand side:
\[ n^2 + 2n + 1 = (n+1)^2 \]
Rearrange the result, and that's our formula back again.  We have proved the inductive step.  

To finish, note that the base case is simply
\[ 1 = 1^2 \]
$\square$

\subsection*{proof of induction}

According to Hamming, if you are not convinced by the ladder analogy, here is another proof that induction works:

\begin{quote}Suppose the statement is not true for every positive (non-negative) integer.  Then there are some false cases.  Consider the set for which the statement is false.  \emph{If} this is a non-empty set, then it would have a least integer, which is $m$.  Now consider the preceeding case, which is $m - 1$.  This $(m-1)$th case must be true by definition, and we know that there is such a case because as a basis for the induction we showed that there was at least one true case.  We now apply the step forward, starting from this true case $m-1$, and conclude that the next case, case $m$, must be true.  But we assumed that it was \emph{false}!  A contradiction. \end{quote}

Therefore, there are no false cases.

$\square$

\end{document}  