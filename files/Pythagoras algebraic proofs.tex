\documentclass[11pt, oneside]{article} 
\usepackage{geometry}
\geometry{letterpaper} 
\usepackage{graphicx}
	
\usepackage{amssymb}
\usepackage{amsmath}
\usepackage{parskip}
\usepackage{color}
\usepackage{hyperref}

\graphicspath{{/Users/telliott/Github/calculus_book/png/}}
% \begin{center} \includegraphics [scale=0.4] {gauss3.png} \end{center}

\title{Pythagorean Algebraic Proofs}
\date{}

\begin{document}
\maketitle
\Large

\label{sec:pythagoras_algebraic}

\subsection*{algebraic proofs}

The following proofs are algebraic ones.  Not so pretty, but fast.  

Arrange 4 identical right triangles as shown in the figure below.  The four triangles plus a small central square form a larger quadrilateral which is also a square.

\begin{center} \includegraphics [scale=0.4] {pythagoras5.png} \end{center}

The angles at the corners of the quadrilateral, at the points flanking the hypotenuse $c$, are right angles, because they are formed by addition of two complementary angles of congruent right triangles.  Since the quadrilateral has four internal right angles and equal length sides, it is a square.

Now just calculate the area of the parts.  We have four identical right triangles with sides $a$ and $b$, plus the central square with sides $b-a$.  The area is 
\[ A = 4 \cdot \frac{1}{2}ab + (b - a)^2 \]
\[ = b^2 + a^2 \]

But the area is also the square of side $c$.  

$\square$

We have used various properties proved earlier, e.g. that the sum of the angles of any triangle is $180$ degrees.

Here is a very similar proof:

\begin{center} \includegraphics [scale=0.5] {pythagoras6.png} \end{center}

In this figure, the right triangles are aligned so that the big square has sides which combine the lengths $a + b$ and have area $(a + b)^2$.  But we can also calculate the area as the sum of its components, namely, central tilted square plus the four triangles:

\[ (a + b)^2 = c^2 + 4 \cdot \frac{ab}{2} \]
\[ a^2 + b^2 + 2ab =  c^2 + 2ab \]
\[ a^2 + b^2 = c^2 \]

$\square$

For the third algebraic proof, divide a right triangle into two smaller ones by dropping an altitude, which meets the base at a right angle.
\begin{center} \includegraphics [scale=0.5] {right_triangle.png} \end{center}

By complementary angles, these three triangles are all similar (e.g., the angle between sides $b$ and $h$ is equal to that between sides $c$ and $a$).  

So we can construct ratios of sides that are equal.  There are three sets, here is one:
\[ \frac{a}{c} = \frac{h}{b} = \frac{d}{a}  \]

We have above a relationship from which to construct $a^2$.
\[ a^2 = cd \]

Another relationship is:
\[ \frac{b}{c} = \frac{c-d}{b} = \frac{h}{a} \]
giving this for $b^2$
\[ b^2 = c(c-d) \]

Simply adding the two together yields:
\[ a^2 + b^2 = cd + c(c-d) = c^2 \]

Which is what we wanted to prove.  $\square$

There are more than 300 proofs of this theorem, including one by a President of the United States, James A. Garfield.  

Here is his proof:

\includegraphics [scale=0.5] {garfield3.png}
\includegraphics [scale=0.5] {garfield2.png}

Draw a right triangle and a rotated copy as shown.  The angles opposite sides $a$ and $b$ are complementary angles.  So the angle marked with a dot is a right angle, and the triangle with sides labeled $c$ is a right triangle.

The area of the quadrilateral is the product of the side $(a + b)$ and the \emph{average} of $a$ and $b$.  This can be seen intuitively (the halfway point of the solid red line has horizontal dimension $(a+b)/2$.

Or, subtract the area of the triangle with two dotted sides from the quadrilateral that includes it:
\[ A = (a + b)b - \frac{(a+b)(b-a)}{2} \]
\[ = (a + b)(b - \frac{b}{2} + \frac{a}{2}) \]
\[ = (a+b) \cdot \frac{1}{2} (a + b) \]
\[ = \frac{a^2}{2} + ab + \frac{b^2}{2} \]

We can also calculate the area of the quadrilateral as the sum of three triangles:
\[ A = \frac{ab}{2} + \frac{ab}{2} + \frac{c^2}{2} \]
Equate the two and the result follows almost immediately.

$\square$
 
\end{document}